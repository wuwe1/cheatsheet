\section{Perfectly Secret Encryption}

\subsection*{Definition}
key: $k \in \mathcal{K}$ The distribution over $\mathcal{K}$ is 
the one defined by running $Gen$ and taking the output.\\
message: $m \in \mathcal{M}$\\
$c\leftarrow Enc_k(m)$: 
possibly probabilistic process by which 
message m is encrypted using key $k$ to give ciphertext $c$\\
$x \leftarrow S$: uniform selection of $x$ from a set $S$\\
$\mathcal{C}$: set of all possible ciphertexts 
that can be output by $Enc_k(m)$\\\\
let $K$ be a random variable denoting the value of the key output by $Gen$:
for any $k \in \mathcal{K},\ \Pr[K=k]$ denotes the probability that key output
by $Gen$ is equal to $k$\\\\
let $M$ be a random variable denoting the message being encrypted:
$\Pr[M=m]$ denotes the probability that the message takes on the value
$m \in \mathcal{M}$\\\\
$\ast\ $The probability distribution of the message is not 
determined by the encryption scheme itself, but instead reflects the likelihood
 of different messages being sent by the parties using the scheme, as well as an
 adversary’s uncertainty about what will be sent.\\
 $\ast\ $ $K$ and $M$ are assumed to be independent.\\

\subsection*{Perfect secrecy} 
ciphertext should have \emph{no effect} on the adversary’s 
knowledge regarding the actual message that was sent\\
Definition: \emph{An encryption scheme 
\emph{(Gen,Enc,Dec)} with message space $\mathcal{M}$ is perfectly secret 
if for every probability distribution over $\mathcal{M}$, 
every message $m \in \mathcal{M}$, and every 
ciphertext $c \in \mathcal{C}$ for which $\Pr[C = c] > 0$:}
$\Pr[M=m|C=c]=Pr[M=m]$\\\\
Equivalent formulation: Informally, this formulation requires that the
 probability distribution of the ciphertext does not depend on the plaintext,
  i.e., for any two messages $m,m'\in\mathcal{M}$ the distribution of the 
  ciphertext when $m$ is encrypted should be identical to the distribution of
   the ciphertext when $m'$ is encrypted. 
   Formally, for every $m,m'\in\mathcal{M}$, and every $c\in\mathcal{C}$,
    \begin{equation}
        \Pr[Enc_K(m)=c]=\Pr[Enc_k(m')=c]\tag{2.1}
    \end{equation}

$\ast\ $This implies that the ciphertext contains no information about 
the plaintext, and that it is impossible to distinguish an encryption of
 $m$ from an encryption of $m'$, since the distributions over the ciphertext
  are the same in each case.\\

  {\color{forestgreen} LEMMA 2.4 \color{forestgreen}}
 \emph{An encryption scheme \emph{(Gen,Enc,Dec)} with message 
space $\mathcal{M}$ is perfectly secret if and only if Equation (2.1) holds 
for every $m,m'\in\mathcal{M}$ and every $c\in\mathcal{C}$.}\\\\

Perfect (adversarial) indistinguishability: Let $\Pi = (Gen,Enc,Dec)$ with 
message space $\mathcal{M}$. Let $\mathcal{A}$ be an adversary. We define the 
experiment $PrivK^{eav}_{\mathcal{A},\Pi}$ as follows:\\
\begin{enumerate}
  \item $\mathcal{A}$ outputs a pair of messages $m_0,m_1\in\mathcal{M}$.
  \item A key $k$ is generated using $Gen$, and a \emph{uniform} bit 
  $b\in\{ 0,1\}$ is chosen. Ciphertext $c\leftarrow Enc_k(m_b)$ is computed and 
  given to $\mathcal{A}$. We refer to c as the \emph{challenge ciphertext}.
  \item $\mathcal{A}$ outputs a bit $b'$
  \item The output of the experiment is defined to be 1 if $b'=b$, and 0 
  otherwise. 
\end{enumerate}

Definition 2.5: $\Pi=(Gen,Enc,Dec)$ with message space $\mathcal{M}$ is 
perfectly indistinguishable if for every $\mathcal{A}$ it holds that\\
$\Pr[PrivK^{eav}_{\mathcal{A},\Pi}=1]=\frac{1}{2}$\\

\subsection*{The One-Time Pad}
LEMMA 2.6: Encryption scheme $\Pi$ is perfectly secret if 
and only if it is perfectly indistinguishable.\\

{\color{forestgreen} THEOREM 2.9 \color{forestgreen}}
: The one-time pad encryption scheme is perfectly secret.\\


{\color{forestgreen} THEOREM 2.10 \color{forestgreen}}:
 If $(Gen, Enc, Dec)$ is a perfectly secret encryption scheme with 
message space $\mathcal{M}$ and key space $\mathcal{K}$, then 
$|\mathcal{K}|\ge|\mathcal{M}|$.
