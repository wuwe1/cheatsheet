\section{Private-Key Encryption}
\subsection*{Computational Security}
Concrete approach:
\begin{enumerate}
    \item Security is only guaranteed against \emph{efficient} 
    adversaries that run for some feasible amount of time
    \item Adversaries can potentially succeed
     with some \emph{very small probability}
\end{enumerate}

A scheme is $(t,\varepsilon)-secure$ if any adversary running for time at most
 $t$ succeeds in breaking the scheme with probability at most $\varepsilon$.\\

 Asymptotic approach:\\
view the running time of the adversary, as well as its success probability, 
 as functions of the security parameter rather than as concrete numbers:
 \begin{enumerate}
     \item We equate “efficient adversaries” with randomized 
     algorithms running in time polynomial in n.
     \item We equate the notion of “small probabilities of success” 
     with success probabilities smaller than any inverse polynomial in n
 \end{enumerate}
 $PPT$: probabilistic polynomial-time\\
 Asymptotic security: A scheme is \emph{secure} if any ppt adversary succeeds
  in breaking the scheme with at most negligible probability.\\

  DEFINITION 3.7: A \emph{private-key encryption scheme} is a tuple of 
  probabilistic polynomial-time algorithms $(Gen, Enc, Dec)$ such that:
  \begin{enumerate}
      \item The \emph{key-generation algorithm Gen} takes as input $1^n$ and outputs
       a key $k$; we write $k \leftarrow Gen(1^n)$(emphasizing that Gen 
       is a randomized algorithm). We assume w.l.o.g that any
        key $k$ output by $Gen(1^n)$ satisfies $|k| \ge n$.
        \item The \emph{encryption algorithm Enc} takes as input a key $k$ and a plaintext
        message $m \in \{0,1\}\ast$, and outputs a ciphertext $c$. Since $Enc$
        may be randomized, we write this as $c \leftarrow Enc_k(m)$.
        \item The \emph{decryption algorithm Dec} takes as input a key $k$ and a ciphertext
        $c$, and outputs a message $m$ or an error. We assume that $Dec$ is 
        deterministic, and so write $m := Dec_k(c)$ 
        (assuming here that Dec does not return an error). 
        We denote a generic error by the symbol $\bot $.
  \end{enumerate}

It is required that for every $n$, every key $k$ output by $Gen(1^n)$, 
and every $m \in \{0,1\}\ast$, it holds that $Dec_k(Enc_k(m)) = m$.\\

$PrivK^{eav}_{\mathcal{A},\Pi}(n)$: 
\begin{enumerate}
    \item The adversary $\mathcal{A}$ is given input $1^n$, and outputs a
     pair of messages $m_0$, $m_1$ with $|m_0| = |m_1|$.
    \item A key $k$ is generated by running $Gen(1^n)$, and a uniform bit
     $b \in \{0,1\}$ is chosen. Ciphertext $c \leftarrow Enc_k(mb)$ is computed
      and given to $\mathcal{A}$. We refer to $c$ as the challenge ciphertext.
    \item $\mathcal{A}$ outputs a bit $b'$
    \item The output of the experiment is defined to be 1 if $b'=b$, and 0
    otherwise.
\end{enumerate}

DEFINITION 3.8: A private-key encryption scheme $\Pi = (Gen, Enc, Dec)$
 has \emph{indistinguishable encryptions in the presence of an eavesdropper}, or 
 is \emph{EAV-secure}, if for all probabilistic polynomial-time adversaries $\mathcal{A}$
 there is a negligible function \emph{negl} such that, for all $n$,
 $\Pr[PrivK^{eav}_{\mathcal{A},\Pi}(n)=1]\le\frac{1}{2}+negl(n)$\\

DEFINITION 3.9: A private-key encryption scheme $\Pi = (Gen, Enc, Dec)$
has \emph{indistinguishable encryptions in the presence of an eavesdropper}
if for all ppt adversaries $\mathcal{A}$ there is a negligible function $negl$ such that
$\Pr[out_{\mathcal{A}}(PrivK^{eav}_{\mathcal{A},\Pi}(n,0)=1)]
-\Pr[out_{\mathcal{A}}(PrivK^{eav}_{\mathcal{A},\Pi}(n,1)=1)]
\le negl(n)$\\
$\ast\ $Essentially states that no $\mathcal{A}$ can determine whether it is running in
experiment 0 or experiment 1