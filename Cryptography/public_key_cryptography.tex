\section{Public-Key (Asymmetric) Cryptography}

\subsection*{Number theory}
Notation:
\begin{itemize}
    \item $\mathbb{Z}$: set of integers
    \item For integer $n,||n||=\lfloor \log n \rfloor + 1$ is the number of 
    bits to represent $n$
    \item $a|b$: $a$ divides $b$ ($\exists c\in \mathbb{Z}$ s.t. $ac=b$)
    \item $a\nmid b$: $a$ does not divides $b$
    \item if $a|b$ and $a>0,a\notin \{1,b\}$ then $a$ is a factor of $b$
    \item Positive integer $p>1$ is \emph{prime} if it has no factors
    \item By convention, the number 1 is neither prime nor composite
    \item An integer $p>1$ that is not prime is \emph{composite}
\end{itemize}
Fundamental Theorem of Arithmetic:\\
All positive integers $n > 1$ can be expressed \emph{uniquely} (up to ordering) as
 $n=\Pi p_i^{\ell_i}$ for primes $p_i$

\subsection*{Greatest Common Divisor(gcd)}
Definition: For $a,b\in \mathbb{Z},gcd(a,b)=c$ s.t. $c$ is the largest integer
so that $c|a$ and $c|b$\\
Properties of gcd:
\begin{enumerate}
    \item If $a,b\in \mathbb{Z}^+$, there exist $X,Y\in \mathbb{Z}$ such that
    $Xa+Yb=gcd(a,b)$
    \item $gcd(a,b)=gcd(b,[\text{a mod b}])$ if $a,b>1$ such that $b \nmid a$
    \item If $c|ab$ and $gcd(a,c)=1$, then $c|b$
    \item If $p$ is prime, then $p|ab$ implies that $p|a$ or $p|b$
    \item If $a|N,b|N$, and $gcd(a,b)=1$ then $ab|N$
    \item $gcd(0,0)$: undefined
    \item $gcd(b,0)=gcd(0,b)=b$
    \item If $p$ is prime, then $gcd(a,p)$ is either $1$ or $p$
\end{enumerate}
Let $a,b$ be positive integers. Then there exist integers $X, Y$ such that
 $X a + Y b = gcd(a, b)$. Furthermore, $gcd(a, b)$ is the smallest positive
  integer that can be expressed in this way.

\subsection*{Modular Arithmetic}
Notation:
\begin{itemize}
    \item $[\text{a mod N}]$: $a$ modulo $N$
    \item $a=\text{b mod N}$ if $[\text{a mod N}]=[\text{b mod N}]$ 
    we say $a$ is congurent to $\text{b mod N}$
\end{itemize}
Congruence Relation: If $a=\text{a' mod N}$, and $b=\text{b' mod N}$ then,
\begin{itemize}
    \item $a+b=a'+b'\ mod\ N$
    \item $ab=a'b'\ mod\ N$
\end{itemize}
Multiplicative inverse: If for a given integer $b$ there exists an integer $c$ 
such that $bc = \text{1 mod N}$, we say that $b$ is \emph{invertible} 
modulo $N$ and call $c$ a (multiplicative) \emph{inverse} of $b$ modulo $N$\\

Let $b$, $N$ be integers, with $b \ge 1$ and $N > 1$. 
Then $b$ is invertible modulo $N$ if and only if $gcd(b, N ) = 1$.

\subsection*{Group theory}
Definition:
\begin{itemize}
    \item Closure: $\forall g,h, \in G, g\cdot h \in G$
    \item Identity: $\exists \text{element } 1_G \in G$ s.t. $\forall g\in G$,
    $1_G\cdot g=g\cdot 1_G=g$
    \item Inverse: $\forall g\in G,\exists h\in G$ s.t. $g\cdot h=h\cdot g=1_G$
    \item Associativity: $\forall g_1,g_2,g_3\in G,(g_1\cdot g_2)\cdot g_3=g_1\cdot(g_2\cdot g_3)$
\end{itemize}

Properties:
\begin{itemize}
    \item $\forall a,b,c \in G$, if $ac=bc$, then $a=b$
    \item Let $|G|=m$, $\forall g\in G,g^m=1_G$
    \item Let $|G|=m$, then for any $g\in G$ and any $x\in \mathbb{Z}$, 
    $g^x=g^{\text{x mod m}}$
    \item Let $|G|=m$, and let $e > 0 \in \mathbb{Z}$. Define
    $f_e: G \rightarrow G$ by $f_e(g)=g^e$
    \begin{itemize}
        \item If $gcd(e,m)=1$, then $f_e$ is a permutation over $G$
        \item If $d=e^{-1}\ mod\ m$, then $f_d=f_e^{-1}$
    \end{itemize}
\end{itemize}

\subsection*{Cyclic Groups}
Let $G$ be a group such that $|G|=m$
\begin{itemize}
    \item For $g\in G$, define $<g>=\{g^0,g^1,\cdots\}$
    \item order of $g\in G$ is smallest $i\le m$ such that $g^i=1$
    \item $<g>=\{g^0,\cdots,g^{i-1}\}$ is a subgroup of $G$
\end{itemize}
A group $G$ is \emph{cyclic} if $\exists g\in G$ s.t. $\text{order}(g)=|G|$.
i.e. $<g>=G$. $g$ is called the generator of $G$\\

If $|G|$ is prime, then $G$ is cyclic. Moreover, all $g\in G$ except 1 are generators.

$\mathbb{Z}^+_N=\{1,\cdots,N-1\}$\\
$\mathbb{Z}^*_N=\{b\in \{1,\cdots,N-1\}|gcd(b,N)=1\}$\\
$\phi(N)=|\mathbb{Z}^*_N|=(p-1)(q-1)$\\
Let $N=\Pi_{i}p_{i}^{e_i}$, $\phi(N)=\Pi_{i}p_{i}^{e_i-1}(p_i-1)$\\
$\phi(ab)=\phi(a) \times \phi(b)$ iff $a$ and $b$ are co-prime\\
Take $N>1$ and $a\in \mathbb{Z}^*_N$, $a^{\phi(N)}=1\text{ mod }N$

\subsection*{Group Isomorphism}
Groups $G$ and $H$ are \emph{isomorphic}$(G\simeq H)$ if there exists function
$f: G\rightarrow H$ such that:
\begin{itemize}
    \item $f$ is a \emph{bijection} (i.e. one-to-one correspondence)
    \item $\forall g_1,g_2 \in G,\ f(g_1\cdot g_2)=f(g_1)\cdot f(g_2)$
\end{itemize}

\subsection*{Chinese Remainder Theorem}
Let $N=pq$, then $\mathbb{Z}_N \simeq \mathbb{Z}_p \times \mathbb{Z}_q$ and
$\mathbb{Z}_N^* \simeq \mathbb{Z}_p^* \times \mathbb{Z}_q^*$, with Isomorphism
$f(x)=([x\text{ mod }p],[x\text{ mod }q])$